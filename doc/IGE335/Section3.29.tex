\subsection{The {\tt S2M:} module}\label{sect:S2MData}

This module is used to extract macrocoscopic cross-section data from a Matlab-formatted {\sc ascii} file
generated by the SERPENT Monte Carlo code (see Ref.~\citen{serpent}) and to convert it to the {\sc macrolib} format.

\vskip 0.02cm

The calling specifications are:

\begin{DataStructure}{Structure \dstr{S2M:}}
\dusa{MACRO}~\moc{:=}~\moc{S2M:}~\dusa{matlab.m}~\moc{::}~\dstr{S2M\_data} \\
\end{DataStructure}

\noindent where
\begin{ListeDeDescription}{mmmmmmm}

\item[\dusa{MACRO}] {\tt character*12} name of the required {\sc macrolib} (type {\tt L\_MACROLIB}) object that is created by {\tt S2M:}.

\item[\dusa{matlab.m}] {\tt character*12} name of a {\sc ascii} file, open in read-only mode, containing Matlab-formatted SERPENT information.

\item[\dusa{S2M\_data}] input data structure containing specific data (see \Sect{descS2M}).

\end{ListeDeDescription}

\subsubsection{Data input for module {\tt S2M:}}\label{sect:descS2M}

\vskip -0.5cm

\begin{DataStructure}{Structure \dstr{S2M\_data}}
$[$~\moc{EDIT} \dusa{iprint}~$]$ \\
$[$~\moc{IDX} \dusa{idx}~$]$ \\
$[$~\moc{B1} $]$ \\
\moc{;}
\end{DataStructure}

\noindent where
\begin{ListeDeDescription}{mmmmmmmm}

\item[\moc{EDIT}] keyword used to set \dusa{iprint}.

\item[\dusa{iprint}] index used to control the printing in module {\tt S2M:}. =0 for no print; =1 for minimum printing (default value).

\item[\moc{IDX}] keyword used to specify the occurence index of a flux calculation in the SERPENT output file. This index generally refers to the burnup step.

\item[\dusa{idx}] occurence index.

\item[\moc{B1}] keyword used to specify that diffusion coefficients and other fundamental-mode information are to be recovered from the SERPENT output file.

\end{ListeDeDescription}

\eject
