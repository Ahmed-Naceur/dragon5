\subsection{The {\tt PSOUR:} module}\label{sect:PSOURData}

This module is used to set fixed sources in a multi-particle coupled transport problem.

\vskip 0.02cm

The calling specifications are:

\begin{DataStructure}{Structure \dstr{PSOUR:}}
\dusa{SOURCE}~\moc{:=}~\moc{PSOUR:}~$\{$~\dusa{MICRO}~$|$~\dusa{MACRO}~$\}$~\dusa{TRACK}~$[[$~\dusa{FLUX}~$]]$~\moc{::}~\dstr{PSOUR\_data} \\
\end{DataStructure}

\noindent where
\begin{ListeDeDescription}{mmmmmmm}

\item[\dusa{SOURCE}] {\tt character*12} name of a {\sc fixed source} (type {\tt L\_SOURCE}) object open in creation
mode. This object contains a unique direct or adjoint fixed source taking into account scattering transitions from one or many companion particles.

\item[\dusa{MICRO}] {\tt character*12} name of a reference {\sc microlib} (type {\tt L\_LIBRARY}) object open in read-only mode. The information on
the embedded macrolib is used.

\item[\dusa{MACRO}] {\tt character*12} name of a reference {\sc macrolib} (type {\tt L\_MACROLIB}) object open in read-only mode.

\item[\dusa{TRACK}] {\tt character*12} name of a reference {\sc tracking} (type {\tt L\_TRACK}) object, corresponding to {\tt L\_SOURCE} object, open in read-only mode.

\item[\dusa{FLUX}] {\tt character*12} name of a {\sc flux} (type {\tt L\_FLUX}) object corresponding to a companion particle open in read-only mode. The number of {\sc flux} objects on the RHS is equal to the number of companion particles contributing to the fixed source.

\item[\dusa{PSOUR\_data}] input data structure containing specific data (see \Sect{descPSOUR}).

\end{ListeDeDescription}

\subsubsection{Data input for module {\tt PSOUR:}}\label{sect:descPSOUR}

\vskip -0.5cm

\begin{DataStructure}{Structure \dstr{PSOUR\_data}}
$[$~\moc{EDIT} \dusa{iprint}~$]$ \\
$[[$~\moc{PARTICLE} \dusa{htype}~$]]$ \\
{\tt ;}
\end{DataStructure}

\noindent where
\begin{ListeDeDescription}{mmmmmmmm}

\item[\moc{EDIT}] keyword used to set \dusa{iprint}.

\item[\dusa{iprint}] index used to control the printing in module {\tt PSOUR:}. =0 for no print; =1 for minimum printing (default value).

\item[\moc{PARTICLE}] keyword used to specify the transition type recovered from the {\sc macrolib} (primary state of the transition). This keyword is repeated for each type of companion particles, in the same order as the \dusa{FLUX} objects on the RHS.

\item[\dusa{htype}] character*1 name of the companion particle. Usual names are {\tt N}: neutrons, {\tt G}: photons, {\tt B}: electrons,
{\tt C}: positrons and {\tt P}: protons.

\end{ListeDeDescription}

\eject
