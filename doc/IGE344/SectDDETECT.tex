\subsection{Contents of a \dir{detect} data structure}\label{sect:detectdir}

\vskip 0.2cm
The \dir{detect} data structure is used to store detector positions and 
responses. This object has a signature {\tt L\_DETECT}; it is created using
the \moc{DETINI:} module. The information contained in
this data structure can be used and updated in other DONJON modules
which are related to the detectors, namely: \moc{DETECT:} and \moc{DETINI:}
modules.

\subsubsection{The state-vector content}\label{sect:devicestate}
\noindent
The dimensioning parameters $\mathcal{S}_i$, which are stored in the state
vector for this data structure, represent:

\begin{itemize}

\item  The number of energy
  groups. $N_g = \mathcal{S}_1$

\item  The total number of detectors $\sum
  \mathcal{I}_{1} = \mathcal{S}_2$ . 

\item  Flag for hexagonal detector definition  $\mathcal{S}_3
  = 1$ for hexagonal detector definition,
$= 0$ otherwise.

\end{itemize}

The dimensioning parameters for a specific detector type, which are stored
in the vector 

$\mathcal{I}_{i}$, represents:

\begin{itemize}

\item The number of detectors of type \{ name\_type \} $\mathcal{I}_{1}$.

\item The number of delayed responses + 2, $\mathcal{I}_{2}$.

\end{itemize}

\subsubsection{The main \dir{detect} directory}\label{sect:detectdir}
\noindent
The following records and sub-directories will be found on the first level
of \dir{detect} directory:

\begin{DescriptionEnregistrement}{Records and sub-directories in \dir{detect}
data structure}{7cm} 

\CharEnr
{SIGNATURE\blank{3}}{$*12$}
{Signature of the \dir{detect} data structure ($\mathsf{SIGNA}=${\tt L\_DETECT\blank{4}}).}

\IntEnr
{STATE-VECTOR}{$40$}
{Vector describing the various parameters associated with this data structure $\mathcal{S}_i$}

\DirVar
{\listedir{name\_type}}
{Detector type sub-directory contains informations
for each detector of this type.}

\end{DescriptionEnregistrement}

\subsubsection{The \dir{name\_type} sub-directories}\label{sect:nametype}

\noindent
Inside each \dir{name\_type} sub-directory, the following records will be found:

\begin{DescriptionEnregistrement}{Records in \dir{name\_type} sub-directories}
{7.0cm} \label{tabl:tabdet}

\IntEnr
{INFORMATIONS}{$2$}
{Record containing describing the various parameters associated with a
  detector type $\mathcal{I}_{i}$.}

\OptRealEnr
{INV-CONST\blank{3}}{$\mathcal{I}_2 -2$}{}{$s^{-1}$}
{The inverse time constant of the
delayed detector responses.}

\OptRealEnr
{FRACTION\blank{4}}{$\mathcal{I}_2 -1$}{}{}
{The delayed and prompt fractions
of the detector responses.}

\RealEnr
{SPECTRAL\blank{4}}{$\mathcal{I}_2 -1$}{}
{The energy spectrum of the
detector.}

\DirVar
{\listedir{name\_detect}}
{Detector information sub-directory}


\end{DescriptionEnregistrement}

\subsubsection{The \dir{name\_detect} sub-directories}\label{sect:detector}

\noindent
Inside each \dir{name\_detect} sub-directory, the following records will be found:

\begin{DescriptionEnregistrement}{Records in \dir{name\_detect} sub-directories}
{7.0cm} \label{tabl:tabrodgroup}

\OptIntEnr
{NHEX\blank{8}}{$nhex$}{$hex=1$}
{The numbers of affected hexagons in the first X-Y plane.}

\RealEnr
{POSITION\blank{4}}{$6$}{cm}
{The coordinates of the 
detector.}

\RealEnr
{RESPON\blank{6}}{$\mathcal{I}_2$}{}
{The responses of the detector.}

\end{DescriptionEnregistrement}
\clearpage
