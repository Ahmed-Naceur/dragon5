\subsection{The \moc{NEWMAC:} module}\label{sect:newmac}

\vskip 0.2cm
The \moc{NEWMAC:} module is used to create a complete \dds{macrolib}
with respect to the devices parameters. The resulting \dds{macrolib} will
contain the exact properties for every material region, over the whole
mesh-splitted reactor geometry. The material properties of each region are
recomputed with respect to the actual position of each rod-type and if present
lzc-type device. The computing algorithm is based on the determination of the
volumic fraction occupied by each device; the incremental cross sections are then
adjusted, accordingly. Note that the \moc{NEWMAC:} module must be executed
each time the devices positions are modified from the previously computed ones.\\

\noindent
The \moc{NEWMAC:} module specification is:

\begin{DataStructure}{Structure \moc{NEWMAC:}}
\dusa{MACRO3} \dusa{MATEX} \moc{:=} \moc{NEWMAC:}
\dusa{MATEX} \dusa{MACRO2} \dusa{DEVICE}
\moc{::} $[$ \moc{EDIT} \dusa{iprint} $]~[$ \moc{XFAC} \dusa{xfac} $]$ ;
\end{DataStructure}

\noindent where
\begin{ListeDeDescription}{mmmmmmmm}

\item[\dusa{MACRO3}] \texttt{character*12} name of the \dds{macrolib}
to be created by the module. It will contain the updated properties of each
material region with respect to the current position of each device.

\item[\dusa{MATEX}] \texttt{character*12} name of the \dds{matex} object,
containing the complete reactor material index including devices. \dusa{MATEX}
must be specified in the modification mode; it will store the updated h-factors,
computed per each fuel region with respect to the devices positions.

\item[\dusa{MACRO2}] \texttt{character*12} name of the read-only extended
\dds{macrolib}, previously created by the \moc{MACINI:} module.

\item[\dusa{DEVICE}] \texttt{character*12} name of the read-only
\dds{device} object containing the devices information and parameters.

\item[\moc{EDIT}] keyword used to set \dusa{iprint}.

\item[\dusa{iprint}] integer index used to control the printing on screen: = 0
for no print; = 1 for minimum printing; larger values produce increasing amounts
of output. The default value is \dusa{iprint} = 1.

\item[\moc{XFAC}] keyword used to specify the number of cells on which incremental cross sections were computed 
in the supercell code.

\item[\dusa{xfac}] corrective factor for delta sigmas (real number). For DRAGON code, \dusa{xfac} is generally set to 2.0 and, for MULTICELL 
code, set to 1.0 . The default value is 2.0. 

\end{ListeDeDescription}
\clearpage
